%% LyX 2.3.7 created this file.  For more info, see http://www.lyx.org/.
%% Do not edit unless you really know what you are doing.
\documentclass[12pt,hyperfootnotes=false]{article}
\usepackage[latin9]{inputenc}
\usepackage{geometry}
\geometry{verbose,tmargin=1in,bmargin=1in,lmargin=1in,rmargin=1in}
\usepackage{float}
\usepackage{amsmath}
\usepackage{amsthm}
\usepackage{amssymb}
\usepackage{graphicx}
\usepackage{setspace}
\usepackage{booktabs}

\usepackage[authoryear]{natbib}
\onehalfspacing

\makeatletter

%%%%%%%%%%%%%%%%%%%%%%%%%%%%%% LyX specific LaTeX commands.
%% Because html converters don't know tabularnewline
\providecommand{\tabularnewline}{\\}

%%%%%%%%%%%%%%%%%%%%%%%%%%%%%% User specified LaTeX commands.
\usepackage{amsthm}
\usepackage{bm}
\usepackage{setspace}
\usepackage{sectsty}

\usepackage{datetime}
\usepackage{pdflscape}
\usepackage[english]{babel}
\usepackage[small]{caption}
\usepackage[bottom,hang,flushmargin]{footmisc}

\DeclareMathOperator{\sgn}{sgn}
\DeclareMathOperator{\Var}{Var}
\DeclareMathOperator{\Corr}{Corr}
\DeclareMathOperator{\Cov}{Cov}
\DeclareMathOperator{\E}{E}
\DeclareMathOperator{\logit}{logit}
\DeclareMathOperator{\I}{I}

\sectionfont{\noindent\normalfont\large\bf}
\subsectionfont{\noindent\normalfont\normalsize\bf}
\subsubsectionfont{\noindent\normalfont\it}

\pdfminorversion=4

\makeatother

\begin{document}
\title{\noindent \textbf{Econ 220A Homework 1}}
\author{\noindent Shiqi Yang,\textbf{ }\textit{UC Berkeley}\textbf{}\thanks{E-mail:\ shiqiy@berkeley.edu
}\textbf{}\\
}
\date{\today}
\maketitle

\begin{spacing}{1.4}

\section{Logit Demand Model with Exogenous Prices}

\subsection*{1.1}

After normalizing calories, added sugar, and protein by package size, we construct a summary table of products available in city 1 during period 1, see below.

\begin{table}[htbp]
\centering
\begin{tabular}{lccccccc}
\toprule
ID & Name & Price & Share & Size & Kcal/g & Sugar/g & Protein/g \\
\midrule
1 & Yoplait & 0.794 & 0.166 & 170 & 0.882 & 0.076 & 0.035 \\
2 & Chobani & 1.144 & 0.164 & 150 & 0.533 & 0.000 & 0.093 \\
3 & Dannon & 1.141 & 0.128 & 150 & 0.667 & 0.000 & 0.100 \\
4 & Stonyfield Farm & 0.904 & 0.109 & 170 & 0.882 & 0.065 & 0.029 \\
5 & Activia & 0.492 & 0.138 & 113 & 0.796 & 0.071 & 0.035 \\
\bottomrule
\end{tabular}

\caption{Summary of Yogurt Products in City 1, Period 1}
\label{tab:yoghurt_summary}
\end{table}


From this table, we find that:

The two most expensive products are Chobani (price = 1.144) and Dannon (price = 1.141).

The two products with the largest market shares are Yoplait (market share = 16.6 \%) and Chobani (market share = 16.4 \%).


\subsection*{1.2}

Let utility be
\[
u_{ijct} = -\alpha p_{jct} + \beta X_{j} + \xi_{jct} + \varepsilon_{ijct},
\]
with $\varepsilon_{ijct}\stackrel{iid}{\sim} \text{EV Type I}$ and the outside option
normalized to $u_{i0ct}=\varepsilon_{i0ct}$. Define the mean utility
\[
\delta_{jct}\equiv -\alpha p_{jct} + \beta X_{j} + \xi_{jct}.
\]

Under the multinomial logit, consumer $i$ chooses product $j$ iff $u_{ijct}\ge u_{ikct}$ for all $k$.
Aggregating over consumers yields the market share of product $j$ in market $(c,t)$:
\[
s_{jct} \;=\; 
\frac{\exp(\delta_{jct})}{1+\sum_{k=1}^{jct}\exp(\delta_{kct})},
\qquad
s_{0ct} \;=\;
\frac{1}{1+\sum_{k=1}^{jct}\exp(\delta_{kct})}.
\]

Taking the ratio $s_{jct}/s_{0ct}$ and logs gives the standard logit inversion:
\[
\log s_{jct} - \log s_{0ct}
\;=\;
\delta_{jct}
\;=\;
-\alpha p_{jct} + \beta X_{j} + \xi_{jct}.
\]

Hence, for each market $(c,t)$ the demand system consists of $jct$ equations,
each depending only on product $j$'s own price and characteristics:
\[
\boxed{\;
\log s_{jct} - \log s_{0ct}
\;=\;
-\alpha p_{jct} + \beta X_{j} + \xi_{jct},
\qquad j=1,\dots,jct.
\;}
\]

\subsection*{1.3}

Under the assumption $\mathbb{E}[\xi_{jct}\mid p_{jct},X_j]=0$, we estimate the
inverted logit demand equation
\[
\log s_{jct}-\log s_{0ct}
= -\alpha\, p_{jct} + \beta_1\text{weight}_j + \beta_2\text{cal/g}_j
  + \beta_3\text{sugar/g}_j + \beta_4\text{protein/g}_j + \xi_{jct}.
\]
We compute $s_{0ct}$ as one minus the sum of inside shares in each market $(c,t)$. The results are presented in \ref{tab:q3_ols_ehw}.

\begin{table}[H]
\centering
\begin{tabular}{lcccc}
\toprule
Variable & Coef. & Std. Err. (HC1) & t & p-value \\
\midrule
Price ($) & -0.8643*** & 0.0196 & -44.05 & 0.000 \\
Package size (g) & 0.0065*** & 0.0002 & 38.73 & 0.000 \\
Calories per g & -1.3305*** & 0.0395 & -33.70 & 0.000 \\
Added sugar per g & 0.0037 & 0.2785 & 0.01 & 0.989 \\
Protein per g & 0.1060 & 0.2103 & 0.50 & 0.614 \\
\bottomrule
\end{tabular}
\caption{OLS estimation of the inverted logit demand (EHW robust SE).}
\label{tab:q3_ols_ehw}
\\end{table}



\subsection*{1.4}

We have

Own-price elasticity:
   \[
   \varepsilon_{jj}
   = \frac{\partial s_j}{\partial p_j}\cdot \frac{p_j}{s_j}
   = -\alpha p_j (1-s_j).
   \]

Cross-price elasticity:
   \[
   \varepsilon_{jm}
   = \frac{\partial s_j}{\partial p_m}\cdot \frac{p_m}{s_j}
   = \alpha p_m s_m.
   \]

Outside option (j=0):
   \[
   s_0 = 1 - \sum_{j=1}^J s_j, \qquad
   \varepsilon_{0m}
   = \frac{\partial s_0}{\partial p_m}\cdot \frac{p_m}{s_0}
   = \alpha p_m s_m \cdot \frac{s_0}{s_0} = \alpha p_m s_m.
   \]

Table \ref{tab:q4_elasticity_city1p1} reports the own price and cross-price elasticity in City 1, Period 1.

\begin{table}[H]
\centering
\caption{Own- and cross-price elasticities in City 1, Period 1 (logit).}
\label{tab:q4_elasticity_city1p1}
\begin{tabular}{lrrrrr}
\toprule
 & Yoplait & Chobani & Dannon & Stonyfield Farm & Activia \\
\midrule
Outside Option & 0.1138 & 0.1618 & 0.1264 & 0.0849 & 0.0586 \\
Yoplait & -0.5728 & 0.1618 & 0.1264 & 0.0849 & 0.0586 \\
Chobani & 0.1138 & -0.8267 & 0.1264 & 0.0849 & 0.0586 \\
Dannon & 0.1138 & 0.1618 & -0.8598 & 0.0849 & 0.0586 \\
Stonyfield Farm & 0.1138 & 0.1618 & 0.1264 & -0.6962 & 0.0586 \\
Activia & 0.1138 & 0.1618 & 0.1264 & 0.0849 & -0.3665 \\
\bottomrule
\end{tabular}
\end{table}



\subsection*{1.5}
\input{Hw1/repo/analysis/output/q5_diversion_city1p1}

\subsection*{1.6}

For a single-product firm $j$ in market $(c,t)$, profit is
\[
\Pi_{jct} = M_{ct}\,(p_{jct} - c_{jct})\, s_{jct}(p_{jct}),
\]
where $s_{jct}$ follows the logit demand with
$\log s_{jct}-\log s_{0ct} = -\alpha p_{jct} + \beta'X_j + \xi_{jct}$.
The first-order condition is
\[
0 = \frac{\partial \Pi_{jct}}{\partial p_{jct}}
= s_{jct}(p) + (p_{jct}-c_{jct}) \frac{\partial s_{jct}(p)}{\partial p_{jct}}.
\]
Using the logit derivative
$\frac{\partial s_{jct}}{\partial p_{jct}} = -\alpha\, s_{jct}(1-s_{jct})$,
we obtain the markup:
\[
p_{jct} - c_{jct} = \frac{1}{\alpha(1-s_{jct})},
\qquad\text{so}\qquad
c_{jct} = p_{jct} - \frac{1}{\alpha(1-s_{jct})}.
\]


\subsection*{1.7}

Using the single-product markup formula from 1.6, we have
\[
c_{jct} = p_{jct} - \frac{1}{\alpha(1-s_{jct})}
\]
Plugging in our estimates ($\hat\alpha\simeq 0.8643$) and the observed
prices and shares in city 1, period 1, we obtain:

\begin{table}[H]
\centering
\caption{Recovered marginal costs (City 1, Period 1).}
\label{tab:q7_mc_c1t1}
\begin{tabular}{lcccc}
\toprule
Product & Price & Share & $1/\{\hat\alpha(1-s)\}$ & Marginal cost $c_{jct}$ \\
\midrule
Yoplait & 0.794 & 0.166 & 1.387 & -0.593 \\
Chobani & 1.144 & 0.164 & 1.383 & -0.240 \\
Dannon & 1.141 & 0.128 & 1.327 & -0.186 \\
Stonyfield Farm & 0.904 & 0.109 & 1.298 & -0.394 \\
Activia & 0.492 & 0.138 & 1.342 & -0.850 \\
\bottomrule
\end{tabular}
\end{table}


Answer: Not all costs are greater than zero; in fact, all recovered costs are
negative in this simple OLS logit setup. 

\subsection*{1.8}

In the inverted-logit equation
\(
\log s_{jct}-\log s_{0ct}=-\alpha p_{jct}+\beta'X_j+\xi_{jct},
\)
prices are typically \emph{endogenous}: positive demand shocks $\xi_{jct}$
(brand quality, promotions, shelf placement, etc.) induce firms to raise prices,
so $\mathrm{Cov}(p_{jct},\xi_{jct})>0$.
If we estimate by OLS imposing $\mathbb{E}[\xi\mid p,X]=0$,
the price coefficient $-\alpha$ is biased upward (less negative), hence $\hat\alpha$
is biased \emph{downward}. Because the implied markup is
\(
p_j-c_j=\frac{1}{\alpha(1-s_j)},
\)
a downward-biased $\hat\alpha$ yields \emph{inflated markups} and thus
\emph{understated marginal costs}, often even negative.
Therefore, recovering reasonable costs requires valid cost shifters and
instrumental variables (e.g., rivals' characteristics, distances, fuel costs),
or a structural supply system with appropriate instruments and ownership.

\end{spacing}
\end{document}
