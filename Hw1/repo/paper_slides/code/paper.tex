%% LyX 2.3.7 created this file.  For more info, see http://www.lyx.org/.
%% Do not edit unless you really know what you are doing.
\documentclass[12pt,hyperfootnotes=false]{article}
\usepackage[latin9]{inputenc}
\usepackage{geometry}
\geometry{verbose,tmargin=1in,bmargin=1in,lmargin=1in,rmargin=1in}
\usepackage{float}
\usepackage{amsmath}
\usepackage{amsthm}
\usepackage{amssymb}
\usepackage{graphicx}
\usepackage{setspace}
\usepackage{booktabs}

\usepackage[authoryear]{natbib}
\onehalfspacing

\makeatletter

%%%%%%%%%%%%%%%%%%%%%%%%%%%%%% LyX specific LaTeX commands.
%% Because html converters don't know tabularnewline
\providecommand{\tabularnewline}{\\}

%%%%%%%%%%%%%%%%%%%%%%%%%%%%%% User specified LaTeX commands.
\usepackage{amsthm}
\usepackage{bm}
\usepackage{setspace}
\usepackage{sectsty}

\usepackage{datetime}
\usepackage{pdflscape}
\usepackage[english]{babel}
\usepackage[small]{caption}
\usepackage[bottom,hang,flushmargin]{footmisc}

\DeclareMathOperator{\sgn}{sgn}
\DeclareMathOperator{\Var}{Var}
\DeclareMathOperator{\Corr}{Corr}
\DeclareMathOperator{\Cov}{Cov}
\DeclareMathOperator{\E}{E}
\DeclareMathOperator{\logit}{logit}
\DeclareMathOperator{\I}{I}

\sectionfont{\noindent\normalfont\large\bf}
\subsectionfont{\noindent\normalfont\normalsize\bf}
\subsubsectionfont{\noindent\normalfont\it}

\pdfminorversion=4

\makeatother

\begin{document}
\title{\noindent \textbf{Econ 220A Homework 1}}
\author{\noindent Shiqi Yang,\textbf{ }\textit{UC Berkeley}\textbf{}\thanks{E-mail:\ shiqiy@berkeley.edu
}\textbf{}\\
}
\date{\today}
\maketitle



\paragraph{Question 3. Analytical expression for $\partial \delta(\sigma)/\partial \sigma$.}

Let $x_j = X_j^s - \bar X^s$ denote the demeaned product characteristic that interacts with
the random coefficient, and let $v_i$ denote the individual draw from a standard normal
distribution. For individual $i$, the deterministic utility component is
\[
\mu_{ij}(\delta,\sigma;v_i) = \delta_j + \sigma v_i x_j,
\]
and the corresponding choice probability is
\[
s_{ij}(\delta,\sigma;v_i)
  = \frac{\exp\{\mu_{ij}(\delta,\sigma;v_i)\}}
         {1 + \sum_{m=1}^{J} \exp\{\mu_{im}(\delta,\sigma;v_i)\}}.
\]
Aggregate market shares are obtained by integrating over individuals:
\[
s_j(\delta,\sigma)
  = \mathbb{E}_i[s_{ij}(\delta,\sigma;v_i)]
  \approx \frac{1}{N} \sum_{i=1}^{N} s_{ij}(\delta,\sigma;v_i).
\]

Observed market shares $s^{\text{obs}}$ satisfy the system of equations
\[
s(\delta(\sigma),\sigma) = s^{\text{obs}}.
\]
Define $F(\delta,\sigma) = s(\delta,\sigma) - s^{\text{obs}} = 0$.
By the Implicit Function Theorem,
\[
\frac{\partial \delta}{\partial \sigma}
  = -\left(\frac{\partial s}{\partial \delta}\right)^{-1}
    \left(\frac{\partial s}{\partial \sigma}\right).
\]
Hence, we need explicit expressions for
$\partial s/\partial \delta$ and $\partial s/\partial \sigma$.

\vspace{0.5em}
\noindent\textbf{Derivative with respect to $\delta$.}
For each draw $i$, define $s_i = (s_{i1},\ldots,s_{iJ})^\top$ and
$\mu_i = (\mu_{i1},\ldots,\mu_{iJ})^\top$.
The standard logit gradient with respect to $\mu_i$ is
\[
\frac{\partial s_i}{\partial \mu_i}
  = \operatorname{diag}(s_i) - s_i s_i^\top.
\]
Since $\partial \mu_i / \partial \delta = I_J$, it follows that
\[
\frac{\partial s_i}{\partial \delta}
  = \operatorname{diag}(s_i) - s_i s_i^\top.
\]
Taking expectations across the individual draws gives
\[
\frac{\partial s}{\partial \delta}
  = \mathbb{E}_i[\operatorname{diag}(s_i) - s_i s_i^\top]
  \;\equiv\; \mathbf{J}.
\]

\vspace{0.5em}
\noindent\textbf{Derivative with respect to $\sigma$.}
For individual $i$,
\[
\frac{\partial \mu_{ij}}{\partial \sigma} = v_i x_j.
\]
Using the chain rule,
\[
\left(\frac{\partial s_i}{\partial \sigma}\right)_j
  = \sum_{m=1}^{J}
     \frac{\partial s_{ij}}{\partial \mu_{im}}
     \frac{\partial \mu_{im}}{\partial \sigma}
  = \sum_{m=1}^{J}
     [s_{ij}(\mathbf{1}\{j=m\}-s_{im})](v_i x_m)
  = v_i s_{ij}\!\left(x_j - \sum_{m=1}^{J}s_{im}x_m\right).
\]
In vector form,
\[
\frac{\partial s_i}{\partial \sigma}
  = v_i\, [\, s_i \odot (x - \mathbf{1}(s_i^\top x))\,],
\]
where $\odot$ denotes elementwise (Hadamard) multiplication.
Aggregating over individuals gives
\[
\frac{\partial s}{\partial \sigma}
  = \mathbb{E}_i
     [\,v_i\, s_i \odot (x - \mathbf{1}(s_i^\top x))\,]
  \;\equiv\; g_\sigma.
\]

\vspace{0.5em}
\noindent\textbf{Final expression.}
Combining the two components,
\[
\boxed{
\frac{\partial \delta(\sigma)}{\partial \sigma}
  = -\,\mathbf{J}^{-1}\, g_\sigma,
}
\]
where
\[
\mathbf{J}
  = \mathbb{E}_i[\operatorname{diag}(s_i) - s_i s_i^\top],
  \qquad
(g_\sigma)_j
  = \mathbb{E}_i[v_i s_{ij}(x_j - s_i^\top x)].
\]
At $\sigma=0$, $s_i$ is identical across $i$, and if the simulation draws
are centered so that $\mathbb{E}[v_i]=0$, then $g_\sigma = 0$ and hence
$\partial \delta / \partial \sigma |_{\sigma=0} = 0$.
This confirms that when the random coefficient is absent, small changes in
$\sigma$ do not alter the mean utility required to match observed market shares.


\paragraph{Question 5. Cost-shifter instrument and identification.}

We construct a cost-shifter instrument following Problem Set~1:
\[
Z^{CS}_{jct} = \text{distance}_{jc} \times \text{diesel}_t,
\]
where $\text{distance}_{jc}$ is the distance between city $c$ and product $j$'s distribution center, and $\text{diesel}_t$ is the diesel price in period $t$. This variable affects prices through marginal costs but is excluded from the demand equation. The final instrument matrix is
\[
Z = [Z^{CS},\ \text{product FE},\ \text{city FE},\ \text{time FE}],
\quad W = (Z'Z)^{-1}.
\]

For a given $\sigma$ and simulated heterogeneity draws $v_i$, we compute $\delta(\sigma)$ via the Berry contraction:
\[
\delta^{(n+1)} = \delta^{(n)} + [\log s^{obs} - \log s(\delta^{(n)},\sigma)].
\]
We then estimate the linear parameters by IV:
\[
\hat{\beta} = ((X'ZWZ'X)^{-1}X'ZWZ'\delta),
\quad \hat{\xi} = \delta - X\hat{\beta}.
\]
The GMM moments and objective function are:
\[
m = \frac{1}{n}Z'\hat{\xi}, \qquad G(\sigma) = 10^6\, m'Wm.
\]

Evaluating at $\sigma = 0$ and $\sigma = 10$ gives nearly identical values:
\[
G(0) = 1.45\times10^{-24}, \qquad G(10) = 1.44\times10^{-24},
\]
and similar price coefficients ($\hat{\alpha}\approx -2.44$). 

The objective function is essentially flat in $\sigma$, implying that the dispersion parameter is weakly identified. The cost-shifter instrument moves prices but carries little information about consumer heterogeneity. Consequently, $G(\sigma)$ provides almost no curvature, illustrating the identification problem emphasized in the assignment.

\end{document}
