%% LyX 2.3.7 created this file.  For more info, see http://www.lyx.org/.
%% Do not edit unless you really know what you are doing.
\documentclass[12pt,hyperfootnotes=false]{article}
\usepackage[latin9]{inputenc}
\usepackage{geometry}
\geometry{verbose,tmargin=1in,bmargin=1in,lmargin=1in,rmargin=1in}
\usepackage{float}
\usepackage{amsmath}
\usepackage{amsthm}
\usepackage{amssymb}
\usepackage{graphicx}
\usepackage{setspace}
\usepackage{booktabs}

\usepackage[authoryear]{natbib}
\onehalfspacing

\makeatletter

%%%%%%%%%%%%%%%%%%%%%%%%%%%%%% LyX specific LaTeX commands.
%% Because html converters don't know tabularnewline
\providecommand{\tabularnewline}{\\}

%%%%%%%%%%%%%%%%%%%%%%%%%%%%%% User specified LaTeX commands.
\usepackage{amsthm}
\usepackage{bm}
\usepackage{setspace}
\usepackage{sectsty}

\usepackage{datetime}
\usepackage{pdflscape}
\usepackage[english]{babel}
\usepackage[small]{caption}
\usepackage[bottom,hang,flushmargin]{footmisc}

\DeclareMathOperator{\sgn}{sgn}
\DeclareMathOperator{\Var}{Var}
\DeclareMathOperator{\Corr}{Corr}
\DeclareMathOperator{\Cov}{Cov}
\DeclareMathOperator{\E}{E}
\DeclareMathOperator{\logit}{logit}
\DeclareMathOperator{\I}{I}

\sectionfont{\noindent\normalfont\large\bf}
\subsectionfont{\noindent\normalfont\normalsize\bf}
\subsubsectionfont{\noindent\normalfont\it}

\pdfminorversion=4

\makeatother

\begin{document}
\title{\noindent \textbf{Econ 220A Homework 1}}
\author{\noindent Shiqi Yang,\textbf{ }\textit{UC Berkeley}\textbf{}\thanks{E-mail:\ shiqiy@berkeley.edu
}\textbf{}\\
}
\date{\today}
\maketitle

\begin{spacing}{1.4}

\section{Logit Demand Model with exogenous prices}

\subsection*{Question 1}

After normalizing calories, added sugar, and protein by package size, we construct a summary table of products available in city 1 during period 1, see below.

\begin{table}[htbp]
\centering
\caption{Summary of Yogurt Products in City 1, Period 1}
\begin{tabular}{lccccccc}
\toprule
ID & Name & Price & Share & Size & Kcal/g & Sugar/g & Protein/g \\
\midrule
1 & Yoplait & 0.794 & 0.166 & 170 & 0.882 & 0.076 & 0.035 \\
2 & Chobani & 1.144 & 0.164 & 150 & 0.533 & 0.000 & 0.093 \\
3 & Dannon & 1.141 & 0.128 & 150 & 0.667 & 0.000 & 0.100 \\
4 & Stonyfield Farm & 0.904 & 0.109 & 170 & 0.882 & 0.065 & 0.029 \\
5 & Activia & 0.492 & 0.138 & 113 & 0.796 & 0.071 & 0.035 \\
\bottomrule
\end{tabular}

\label{tab:yoghurt_summary}
\end{table}


From this table, we find that:

The two most expensive products are Chobani (price = 1.144) and Dannon (price = 1.141).

The two products with the largest market shares are Yoplait (market share = 16.6 \%) and Chobani (market share = 16.4 \%).


\subsection*{Question 2}

Let utility be
\[
u_{ijct} = -\alpha p_{jct} + \beta X_{j} + \xi_{jct} + \varepsilon_{ijct},
\]
with $\varepsilon_{ijct}\stackrel{iid}{\sim} \text{EV Type I}$ and the outside option
normalized to $u_{i0ct}=\varepsilon_{i0ct}$. Define the mean utility
\[
\delta_{jct}\equiv -\alpha p_{jct} + \beta X_{j} + \xi_{jct}.
\]

Under the multinomial logit, consumer $i$ chooses product $j$ iff $u_{ijct}\ge u_{ikct}$ for all $k$.
Aggregating over consumers yields the market share of product $j$ in market $(c,t)$:
\[
s_{jct} \;=\; 
\frac{\exp(\delta_{jct})}{1+\sum_{k=1}^{jct}\exp(\delta_{kct})},
\qquad
s_{0ct} \;=\;
\frac{1}{1+\sum_{k=1}^{jct}\exp(\delta_{kct})}.
\]

Taking the ratio $s_{jct}/s_{0ct}$ and logs gives the standard logit inversion:
\[
\log s_{jct} - \log s_{0ct}
\;=\;
\delta_{jct}
\;=\;
-\alpha p_{jct} + \beta X_{j} + \xi_{jct}.
\]

Hence, for each market $(c,t)$ the demand system consists of $jct$ equations,
each depending only on product $j$'s own price and characteristics:
\[
\boxed{\;
\log s_{jct} - \log s_{0ct}
\;=\;
-\alpha p_{jct} + \beta X_{j} + \xi_{jct},
\qquad j=1,\dots,jct.
\;}
\]

\subsection*{Question 3}

Under the assumption $\mathbb{E}[\xi_{jct}\mid p_{jct},X_j]=0$, we estimate the
inverted logit demand equation
\[
\log s_{jct}-\log s_{0ct}
= -\alpha\, p_{jct} + \beta_1\text{weight}_j + \beta_2\text{cal/g}_j
  + \beta_3\text{sugar/g}_j + \beta_4\text{protein/g}_j + \xi_{jct}.
\]
We compute $s_{0ct}$ as one minus the sum of inside shares in each market $(c,t)$. The results are presented in \ref{tab:q3_ols_ehw}.

\begin{table}[H]
\centering
\begin{tabular}{lcccc}
\toprule
Variable & Coef. & Std. Err. (HC1) & t & p-value \\
\midrule
Price ($) & -0.864*** & 0.020 & -44.052 & 0.000 \\
Package size (g) & 0.006*** & 0.000 & 38.727 & 0.000 \\
Calories per g & -1.330*** & 0.039 & -33.701 & 0.000 \\
Added sugar per g & 0.004 & 0.279 & 0.013 & 0.989 \\
Protein per g & 0.106 & 0.210 & 0.504 & 0.614 \\
\bottomrule
\end{tabular}
\caption{OLS estimation of the inverted logit demand (EHW robust SE).}
\label{tab:q3_ols_ehw}
\\end{table}



\subsection*{Question 4}

We have

Own-price elasticity:
   \[
   \varepsilon_{jj}
   = \frac{\partial s_j}{\partial p_j}\cdot \frac{p_j}{s_j}
   = -\alpha p_j (1-s_j).
   \]

Cross-price elasticity:
   \[
   \varepsilon_{jm}
   = \frac{\partial s_j}{\partial p_m}\cdot \frac{p_m}{s_j}
   = \alpha p_m s_m.
   \]

Outside option (j=0):
   \[
   s_0 = 1 - \sum_{j=1}^J s_j, \qquad
   \varepsilon_{0m}
   = \frac{\partial s_0}{\partial p_m}\cdot \frac{p_m}{s_0}
   = \alpha p_m s_m \cdot \frac{s_0}{s_0} = \alpha p_m s_m.
   \]

Table \ref{tab:q4_elasticity_city1p1} reports the own price and cross-price elasticity in City 1, Period 1.

\begin{table}[H]
\centering
\caption{Own- and cross-price elasticities in City 1, Period 1 (logit).}
\label{tab:q4_elasticity_city1p1}
\begin{tabular}{lrrrrrr}
\toprule
 & Outside Option & Yoplait & Chobani & Dannon & Stonyfield Farm & Activia \\
\midrule
Outside Option & 0.0000 & 0.1138 & 0.1618 & 0.1264 & 0.0849 & 0.0586 \\
Yoplait & 0.0000 & -0.5728 & 0.1618 & 0.1264 & 0.0849 & 0.0586 \\
Chobani & 0.0000 & 0.1138 & -0.8267 & 0.1264 & 0.0849 & 0.0586 \\
Dannon & 0.0000 & 0.1138 & 0.1618 & -0.8598 & 0.0849 & 0.0586 \\
Stonyfield Farm & 0.0000 & 0.1138 & 0.1618 & 0.1264 & -0.6962 & 0.0586 \\
Activia & 0.0000 & 0.1138 & 0.1618 & 0.1264 & 0.0849 & -0.3665 \\
\bottomrule
\end{tabular}
\end{table}



\subsection*{Question 5}
\begin{table}[H]\centering
\caption{Diversion ratios in City 1, Period 1 (logit).}
\label{tab:q5_diversion}
\begin{tabular}{lrrrrrr}
\toprule
 & Yoplait & Chobani & Dannon & Stonyfield Farm & Activia & Outside option \\
\midrule
Yoplait & 0.0000 & 0.1962 & 0.1537 & 0.1302 & 0.1653 & 0.3546 \\
Chobani & 0.1982 & 0.0000 & 0.1533 & 0.1299 & 0.1649 & 0.3537 \\
Dannon & 0.1901 & 0.1877 & 0.0000 & 0.1246 & 0.1582 & 0.3393 \\
Stonyfield Farm & 0.1860 & 0.1836 & 0.1438 & 0.0000 & 0.1547 & 0.3319 \\
Activia & 0.1923 & 0.1898 & 0.1487 & 0.1260 & 0.0000 & 0.3431 \\
\bottomrule
\end{tabular}
\end{table}


\subsection*{Question 6}

For a single-product firm $j$ in market $(c,t)$, profit is
\[
\Pi_{jct} = M_{ct}\,(p_{jct} - c_{jct})\, s_{jct}(p_{jct}),
\]
where $s_{jct}$ follows the logit demand with
$\log s_{jct}-\log s_{0ct} = -\alpha p_{jct} + \beta'X_j + \xi_{jct}$.
The first-order condition is
\[
0 = \frac{\partial \Pi_{jct}}{\partial p_{jct}}
= s_{jct}(p) + (p_{jct}-c_{jct}) \frac{\partial s_{jct}(p)}{\partial p_{jct}}.
\]
Using the logit derivative
$\frac{\partial s_{jct}}{\partial p_{jct}} = -\alpha\, s_{jct}(1-s_{jct})$,
we obtain the markup:
\[
p_{jct} - c_{jct} = \frac{1}{\alpha(1-s_{jct})},
\qquad\text{so}\qquad
c_{jct} = p_{jct} - \frac{1}{\alpha(1-s_{jct})}.
\]


\subsection*{Question 7}

Using the single-product markup formula from Question 6, we have
\[
c_{jct} = p_{jct} - \frac{1}{\alpha(1-s_{jct})}
\]
Plugging in our estimates ($\hat\alpha\simeq 0.8643$) and the observed
prices and shares in city 1, period 1, we obtain:

\begin{table}[H]
\centering
\caption{Recovered marginal costs (City 1, Period 1).}
\label{tab:q7_mc_c1t1}
\begin{tabular}{lcccc}
\toprule
Product & Price & Share & $1/\{\hat\alpha(1-s)\}$ & Marginal cost $c_{jct}$ \\
\midrule
Yoplait & 0.794 & 0.166 & 1.387 & -0.593 \\
Chobani & 1.144 & 0.164 & 1.383 & -0.240 \\
Dannon & 1.141 & 0.128 & 1.327 & -0.186 \\
Stonyfield Farm & 0.904 & 0.109 & 1.298 & -0.394 \\
Activia & 0.492 & 0.138 & 1.342 & -0.850 \\
\bottomrule
\end{tabular}
\end{table}


Answer: Not all costs are greater than zero; in fact, all recovered costs are
negative in this simple OLS logit setup. 

\subsection*{Question 8}

In the inverted-logit equation
\(
\log s_{jct}-\log s_{0ct}=-\alpha p_{jct}+\beta'X_j+\xi_{jct},
\)
prices are typically endogenous: positive demand shocks $\xi_{jct}$
(brand quality, promotions, shelf placement, etc.) induce firms to raise prices,
so $\mathrm{Cov}(p_{jct},\xi_{jct})>0$.
If we estimate by OLS imposing $\mathbb{E}[\xi\mid p,X]=0$,
the price coefficient $-\alpha$ is biased upward (less negative), hence $\hat\alpha$
is biased downward. Because the implied markup is
\(
p_j-c_j=\frac{1}{\alpha(1-s_j)},
\)
a downward-biased $\hat\alpha$ yields over-estimated markups and thus
under-estimated marginal.

\section{Logit Demand Model with three way fixed effects}

\subsection*{Question 9}
Consider the inverted-logit estimating equation
\[
\log s_{jct}-\log s_{0,ct}
= -\alpha\,p_{jct} \;+\; \tau_j \;+\; \tau_c \;+\; \tau_t \;+\; \varepsilon_{jct},
\]
where $\tau_j$ (product FE), $\tau_c$ (city FE), and $\tau_t$ (period FE) are unrestricted
additive terms. The product FE absorbs time- and market-irrelevant quality/brand effect that
is typically positively correlated with prices (``high-quality--high-price''), the city FE absorbs
location-wide demand and cost shifters, and the period FE absorbs aggregate shocks (inflation,
seasonality, national advertising, etc.). Those fixed effects absorbed a large part of the omitted-variable bias present in the previous OLS without FE.

However, the unobserved, market-specific quality $\xi_{jct}$ (e.g., promotion, shelf placement)
may still be correlated with prices within a market, so three-way FE do not in general solve price
endogeneity. Instruments or cost shifters are still needed for full identification.

\subsection*{Question 10}

We estimate
\[
\log s_{jct}-\log s_{0,ct}
= -\alpha\,p_{jct} + \tau_j + \tau_c + \tau_t + \varepsilon_{jct}
\]
by OLS with EHW (HC1) standard errors and product/city/period fixed effects.
Given the FE estimate $\hat\alpha$, market prices $p_{j11}$ and shares $s_{j11}$, own- and
cross-price elasticities in a multinomial logit are
\[
\varepsilon_{jj,11} = \frac{\partial s_{j11}}{\partial p_{j11}}\frac{p_{j11}}{s_{j11}}
= -\hat\alpha\,p_{j11}\,(1-s_{j11}),
\qquad
\varepsilon_{jk,11}
= \frac{\partial s_{j11}}{\partial p_{k11}}\frac{p_{k11}}{s_{j11}}
= \hat\alpha\,p_{k11}\,s_{k11} \quad (k\neq j).
\]
The diversion ratio from $j$ to $k$ is
\[
D_{j\to k,\,11}
= \frac{-\partial s_{k,11}/\partial p_{j,11}}{-\partial s_{0,11}/\partial p_{j,11}}
= \frac{s_{k,11}}{1-s_{j,11}}.
\]
Relative to the no-FE estimates, $|\hat\alpha|$ typically increases (less attenuation from
uncontrolled quality), so both own- and cross-price elasticities increase in magnitude, while
diversion ratios---which depend only on market shares in MNL---remain unchanged.

The results are presented below.

\begin{table}[H]
\centering
\caption{FE-OLS of the inverted logit demand (HC1; price coefficient only).}
\label{tab:q10_fe_ols_price}
\begin{tabular}{lcccc}
\toprule
Variable & Coef. & Std. Err. (HC1) & t & p-value \\
\midrule
Price & -0.869*** & 0.018 & -48.517 & 0.000 \\
\bottomrule
\end{tabular}
\end{table}



\begin{table}[H]
\centering
\caption{Own- and cross-price elasticities (City 1, Period 1, FE-OLS $\hat\alpha$).}
\label{tab:q10_elast_c1t1_fe}
\begin{tabular}{lrrrrr}
\toprule
 & Yoplait & Chobani & Dannon & Stonyfield Farm & Activia \\
\midrule
Outside Option & 0.1144 & 0.1626 & 0.1271 & 0.0853 & 0.0590 \\
Yoplait & -0.5758 & 0.1626 & 0.1271 & 0.0853 & 0.0590 \\
Chobani & 0.1144 & -0.8311 & 0.1271 & 0.0853 & 0.0590 \\
Dannon & 0.1144 & 0.1626 & -0.8643 & 0.0853 & 0.0590 \\
Stonyfield Farm & 0.1144 & 0.1626 & 0.1271 & -0.6999 & 0.0590 \\
Activia & 0.1144 & 0.1626 & 0.1271 & 0.0853 & -0.3684 \\
\bottomrule
\end{tabular}
\end{table}


\begin{table}[H]
\centering
\caption{Diversion ratios (City 1, Period 1, logit).}
\label{tab:q10_div_c1t1_fe}
\begin{tabular}{lrrrrrr}
\toprule
 & Yoplait & Chobani & Dannon & Stonyfield Farm & Activia & Outside option \\
\midrule
Yoplait & 0.0000 & 0.1962 & 0.1537 & 0.1302 & 0.1653 & 0.3546 \\
Chobani & 0.1982 & 0.0000 & 0.1533 & 0.1299 & 0.1649 & 0.3537 \\
Dannon & 0.1901 & 0.1877 & 0.0000 & 0.1246 & 0.1582 & 0.3393 \\
Stonyfield Farm & 0.1860 & 0.1836 & 0.1438 & 0.0000 & 0.1547 & 0.3319 \\
Activia & 0.1923 & 0.1898 & 0.1487 & 0.1260 & 0.0000 & 0.3431 \\
\bottomrule
\end{tabular}
\end{table}



\subsection*{Question 11}

Under Bertrand competition, the first-order conditions imply
\[
s_{ct} = \Delta_{ct}\,(p_{ct}-c_{ct}),
\quad
\Delta_{ct} = \hat\alpha \,\Omega \odot \big(\mathrm{diag}(s_{ct})-s_{ct}s_{ct}^\top\big),
\]
where $\Omega$ is the ownership matrix and $\odot$ denotes the Hadamard product. Hence
\[
\boxed{ \; c_{ct} = p_{ct} - \Delta_{ct}^{-1}\,s_{ct}\; }.
\]
If each firm sells a single product ($\Omega=I$), this reduces to
\[
p_{jct}-c_{jct}=\frac{s_{jct}}{\hat\alpha\,(1-s_{jct})},
\qquad
c_{jct}=p_{jct}-\frac{s_{jct}}{\hat\alpha\,(1-s_{jct})}.
\]
Using $\hat\alpha$ from the FE regression and the observed prices and shares in city 1 and
period 1, we recover $c_{j11}$.

Adding FE often raises implied marginal costs (markups fall with larger $|\hat\alpha|$), but FE
do not remove within-market price endogeneity, measurement error in prices/shares, potential
multi-product ownership, or misspecification of demand (IIA). Any of these can still yield
implausible costs---including negative values (e.g., for Activia in this dataset). This is why FE
alone may not be sufficient; one typically adds valid cost instruments and, if needed, a random
coefficients demand system.




\section{Logit Demand Model using cost shifters}

\subsection*{Question 12}
We instrument price with a cost shifter: for product $j$ in city $c$ and period $t$,
\[
Z_{jct}= \text{distance}_{cj}\times \text{diesel}_t .
\]
In three-way fixed effects (product, city, period) model, identification of $\alpha$ comes from
within-product price variation across cities/periods induced by $Z_{jct}$, while FE purge most time-invariant quality, city demand, and aggregate shocks. This specification is not equivalent to a model that only includes $X_j$: the FE specification removes many confounds that $X_j$ cannot capture and then uses $Z_{jct}$ to address remaining within-market price endogeneity. Hence IV with FE is more likely to deliver less biased own-price elasticities. Exclusion could fail if distance also shifts demand (e.g., freshness, shelf access) or if distribution-center locations are chosen endogenously.

\subsection*{Question 13}
We estimate
\[
\log s_{jct}-\log s_{0,ct}=-\alpha p_{jct}+\tau_j+\tau_c+\tau_t+\varepsilon_{jct}
\]
by IV with $Z_{jct}$ and EHW (HC1) standard errors. Given $\hat\alpha$, own- and cross-price elasticities in market $(1,1)$ are
\[
\varepsilon_{jj}=-\hat\alpha\,p_j(1-s_j),\qquad
\varepsilon_{jk}=\hat\alpha\,p_k s_k\;(k\neq j),
\]
and diversion ratios are $D_{j\to k}= s_k/(1-s_j)$. Table 9 - 11 reports the results. Relative to OLS or FE-OLS, $|\hat\alpha|$ typically increases (correcting price endogeneity), so elasticities grow in magnitude, while diversion ratios are unchanged because they depend only on shares under MNL.

\begin{table}[H]
\centering
\begin{tabular}{lcccc}
\toprule
Variable & Coef. & Std. Err. (HC1) & t & p-value \\
\midrule
Price & -3.433984 & 0.051552 & -66.611502 & NaN \\
\bottomrule
\end{tabular}
\caption{IV(transport cost) with product/city/period FE (HC1).}
\label{tab:q13_iv_fe_price}
\end{table}

\begin{table}[H]
\centering
\caption{Own- and cross-price elasticities (City 1, Period 1, IV-FE $\hat\alpha$).}
\label{tab:q13_elast_c1t1}
\begin{tabular}{lrrrrr}
\toprule
 & Yoplait & Chobani & Dannon & Stonyfield Farm & Activia \\
\midrule
Outside Option & 0.4522 & 0.6428 & 0.5023 & 0.3372 & 0.2330 \\
Yoplait & -2.2757 & 0.6428 & 0.5023 & 0.3372 & 0.2330 \\
Chobani & 0.4522 & -3.2848 & 0.5023 & 0.3372 & 0.2330 \\
Dannon & 0.4522 & 0.6428 & -3.4160 & 0.3372 & 0.2330 \\
Stonyfield Farm & 0.4522 & 0.6428 & 0.5023 & -2.7663 & 0.2330 \\
Activia & 0.4522 & 0.6428 & 0.5023 & 0.3372 & -1.4562 \\
\bottomrule
\end{tabular}
\end{table}

\begin{table}[H]
\centering
\caption{Diversion ratios (City 1, Period 1).}
\label{tab:q13_div_c1t1}
\begin{tabular}{lrrrrrr}
\toprule
 & Yoplait & Chobani & Dannon & Stonyfield Farm & Activia & Outside option \\
\midrule
Yoplait & 0.0000 & 0.1962 & 0.1537 & 0.1302 & 0.1653 & 0.3546 \\
Chobani & 0.1982 & 0.0000 & 0.1533 & 0.1299 & 0.1649 & 0.3537 \\
Dannon & 0.1901 & 0.1877 & 0.0000 & 0.1246 & 0.1582 & 0.3393 \\
Stonyfield Farm & 0.1860 & 0.1836 & 0.1438 & 0.0000 & 0.1547 & 0.3319 \\
Activia & 0.1923 & 0.1898 & 0.1487 & 0.1260 & 0.0000 & 0.3431 \\
\bottomrule
\end{tabular}
\end{table}


\subsection*{Question 14}
Bertrand FOCs imply $s=\Delta(p-c)$ with
\[
\Delta=\hat\alpha\big(\Omega\odot(\mathrm{diag}(s)-ss^\top)\big).
\]
With single-product firms ($\Omega=I$),
\[
p_j-c_j=\frac{1}{\hat\alpha(1-s_j)},\qquad
c_j=p_j-\frac{1}{\hat\alpha(1-s_j)} .
\]
Using prices and shares in city 1, period 1 with IV-FE $\hat\alpha$, we recover marginal costs. Costs are expected to be higher (markups smaller) than in FE-OLS; all costs are greater than 0.

\begin{table}[H]
\centering
\caption{Recovered marginal costs (City 1, Period 1, IV-FE $\hat\alpha$).}
\label{tab:q14_mc_c1t1}
\begin{tabular}{lcccc}
\toprule
Product & Price & Share & $1/\{\hat\alpha(1-s)\}$ & Marginal cost $c_{jct}$ \\
\midrule
Yoplait & 0.7944 & 0.1658 & 0.3491 & 0.4453 \\
Chobani & 1.1437 & 0.1637 & 0.3482 & 0.7956 \\
Dannon & 1.1410 & 0.1282 & 0.3340 & 0.8070 \\
Stonyfield Farm & 0.9037 & 0.1086 & 0.3267 & 0.5770 \\
Activia & 0.4919 & 0.1379 & 0.3378 & 0.1541 \\
\bottomrule
\end{tabular}
\end{table}


\subsection*{Question 15}
Algorithm to recover equilibrium prices with estimated $\hat\alpha$ and $c$:
\begin{enumerate}\setlength{\itemsep}{0pt}
\item Calibrate $\delta^{obs}_j=\log s_j-\log s_0+\hat\alpha\,p^{obs}_j$.
\item Initialize $p^{(0)}=p^{obs}$. For iteration $t$:
\begin{align*}
\delta^{(t)}&=\delta^{obs}-\hat\alpha p^{(t)},\quad
s^{(t)}=\frac{\exp(\delta^{(t)})}{1+\sum_k \exp(\delta^{(t)}_k)},\\
\Delta^{(t)}&=\hat\alpha\big(\Omega\odot(\mathrm{diag}(s^{(t)})-s^{(t)}s^{(t)\top})\big),\\
p^{(t+1)}&=c+\left(\Delta^{(t)}\right)^{-1}s^{(t)} .
\end{align*}
\item Stop when $\|p^{(t+1)}-p^{(t)}\|_\infty<\varepsilon$.
\end{enumerate}
Using $\Omega=I$ and $c_j=p^{obs}_j-1/[\hat\alpha(1-s_j)]$, the algorithm reproduces the observed prices for city 1, period 1 up to numerical error.

\begin{table}[H]
\centering
\caption{Observed vs. recovered equilibrium prices (City 1, Period 1).}
\begin{tabular}{lccc}
\toprule
Product & Observed price & Recovered price & Abs. diff \\
\midrule
1 & 0.794383 & 0.804723 & 0.010340 \\
2 & 1.143743 & 1.153936 & 0.010193 \\
3 & 1.141041 & 1.148824 & 0.007784 \\
4 & 0.903747 & 0.910244 & 0.006498 \\
5 & 0.491901 & 0.500337 & 0.008436 \\
\bottomrule
\end{tabular}
\label{tab:q15_recover_c1t1}
\end{table}


\subsection*{Question 16}
Counterfactual merger (Chobani + Dannon) with constant marginal costs: set $\Omega$ so that the two products share common ownership and resolve the fixed point above. Prices of the merged products increase because the firm internalizes cannibalization; rivals' prices typically rise as well by strategic complementarity.

\begin{table}[H]
\centering
\caption{{{caption}}}
\begin{tabular}{lcccc}
\toprule
Product & Old price & New price (merge) & Diff \\
\midrule
Yoplait & 0.804723 & 0.809912 & 0.005189 \\
Chobani & 1.153936 & 1.207848 & 0.053912 \\
Dannon & 1.148824 & 1.219309 & 0.070485 \\
Stonyfield Farm & 0.910244 & 0.913530 & 0.003286 \\
Activia & 0.500337 & 0.504588 & 0.004251 \\
\bottomrule
\end{tabular}
\label{tab:q16_merge_c1t1}
\end{table}


\subsection*{Question 17}
Consumer welfare in MNL is the inclusive value
\[
CS(p)=\frac{1}{\hat\alpha}\log\!\Big(1+\sum_j e^{\delta^{obs}_j-\hat\alpha p_j}\Big).
\]
We compute $CS$ per capita with and without the merger using the fixed-point prices from above, and compare the difference to per-capita expenditure $\sum_j p_j s_j$. Consumers are typically worse off after the merger (higher prices reduce the inclusive value); the magnitude of the welfare loss relative to total expenditure is reported below.

\begin{table}[H]
\centering
\caption{{{caption}}}
\begin{tabular}{lcc}
\toprule
Metric & Pre-merger & Post-merger \\
\midrule
Consumer surplus per capita & 0.304994 & 0.282629 \\
Total expenditure per capita & 0.733328 & 0.727484 \\
Difference (post - pre) & -0.022365 & -0.005844 \\
CS change as \% of pre expenditure & -3.049731 & NaN \\
\bottomrule
\end{tabular}
\label{tab:q17_sum_c1t1}
\end{table}



\section{Nested Logit Demand Model}

\subsection*{Question 18}
Under the standard logit, IIA forces substitution to be proportional to market shares. When Chobani and Dannon are both zero-sugar products, a price increase of one should mainly divert demand to the other, but IIA spreads diversion across all inside options and the outside good. This makes the cross-price elasticity between the two zero-sugar products biased downward, so a merger between them looks less harmful: markups and post-merger prices are understated and consumer-welfare losses are overstated less. A nested logit with a ``no-sugar'' nest allows within-nest correlation and concentrates substitution inside the nest, alleviating this bias.

\subsection*{Question 19}
Let $g$ index nests (no-sugar, sugary, outside). With single-level nested logit and correlation parameter $\rho\in[0,1)$, market shares factor as
\[
s_{jct}=s_{g,ct}\,s_{j|g,ct},
\quad
s_{j|g,ct}=\frac{\exp(\delta_{jct}/(1-\rho))}{\sum_{k\in g}\exp(\delta_{kct}/(1-\rho))}.
\]
The estimating equation that depends only on product $j$'s observables and conditional share is
\[
\boxed{\ \log s_{jct}-\log s_{0,ct}
= -\alpha\,p_{jct}+\rho\log s_{j|g,ct}+x_{j}'\beta+\tau_j+\tau_c+\tau_t+\varepsilon_{jct}\ }.
\]

\subsection*{Question 20}
We estimate the nested-logit equation by 2SLS with product/city/period fixed effects and EHW (HC1) standard errors. The instrument for price is ${\rm distance}_{cj}\times {\rm diesel}_t$; the instrument for $\log s_{j|g}$ is the (log) number of products in nest $g$ in market $(c,t)$. The estimates of $\alpha$ and $\rho$ are reported below.
\begin{table}[H]
\centering
\caption{Nested-logit IV with product/city/period FE (HC1).}
\begin{tabular}{lcc}
\toprule
Variable & Coef. & Std. Err. (HC1) \\
\midrule
Price ($\alpha$) & -2.1335 & 0.0171 \\
$\log s_{j|g}$ ($\rho$) & 0.5197 & 0.0050 \\
\bottomrule
\end{tabular}
\label{tab:q20_nl_ivfe}
\end{table}


\subsection*{Question 21}
Own- and cross-price elasticities in market $(1,1)$ follow the nested-logit formulas
\[
\varepsilon_{jj}=-\hat\alpha\,p_j\big[1-\hat\rho(1-s_{j|g})-s_j\big],\qquad
\varepsilon_{jk}=
\begin{cases}
\hat\alpha\,p_k\,[\hat\rho\,s_{k|g}-s_k], & k\in g,\ k\neq j,\\
\hat\alpha\,p_k\,s_k, & k\notin g,
\end{cases}
\]
and diversion ratios are larger within nest than across nests. Tables:
\begin{table}[H]
\centering
\caption{Nested-logit elasticities (City 1, Period 1).}
\label{tab:q21_elast}
\begin{tabular}{lrrrrr}
\toprule
 & Yoplait & Chobani & Dannon & Stonyfield Farm & Activia \\
\midrule
Outside Option & 0.2769 & 0.3936 & 0.3076 & 0.2065 & 0.1427 \\
Yoplait & -0.8622 & 0.3936 & 0.3076 & 0.0599 & 0.0414 \\
Chobani & 0.2769 & -1.4496 & 0.2531 & 0.2065 & 0.1427 \\
Dannon & 0.2769 & 0.3238 & -1.3762 & 0.2065 & 0.1427 \\
Stonyfield Farm & 0.0803 & 0.3936 & 0.3076 & -0.9494 & 0.0414 \\
Activia & 0.0803 & 0.3936 & 0.3076 & 0.0599 & -0.5256 \\
\bottomrule
\end{tabular}
\end{table}

\begin{table}[H]
\centering
\caption{Nested-logit diversion ratios (City 1, Period 1).}
\label{tab:q21_div}
\begin{tabular}{lrrrrrr}
\toprule
 & Yoplait & Chobani & Dannon & Stonyfield Farm & Activia & Outside option \\
\midrule
Yoplait & 0.0000 & -0.3171 & -0.2484 & -0.0610 & -0.0775 & -0.2960 \\
Chobani & -0.2751 & 0.0000 & -0.1750 & -0.1803 & -0.2289 & -0.1408 \\
Dannon & -0.2890 & -0.2348 & 0.0000 & -0.1894 & -0.2405 & -0.0463 \\
Stonyfield Farm & -0.0962 & -0.3276 & -0.2566 & 0.0000 & -0.0801 & -0.2395 \\
Activia & -0.0946 & -0.3221 & -0.2523 & -0.0620 & 0.0000 & -0.2689 \\
\bottomrule
\end{tabular}
\end{table}


\subsection*{Question 22}
With single-product firms, the Bertrand FOCs imply
\[
p_j-c_j=\frac{1}{\hat\alpha\,[1-\hat\rho(1-s_{j|g})-s_j]},\qquad
c_j=p_j-\frac{1}{\hat\alpha\,[1-\hat\rho(1-s_{j|g})-s_j]} .
\]
Using prices and shares in city 1, period 1 with the IV-FE estimates $(\hat\alpha,\hat\rho)$ we obtain:
\begin{table}[H]
\centering
\caption{Nested-logit recovered marginal costs (City 1, Period 1).}
\label{tab:q22_nl_costs}
\begin{tabular}{lcccc}
\toprule
Product & Price & Share & $1/\{\hat\alpha[1-\hat\rho(1-s_{j|g})-s_j]\}$ & Marginal cost $c_j$ \\
\midrule
Yoplait & 0.7944 & 0.1658 & 0.9213 & -0.1269 \\
Chobani & 1.1437 & 0.1637 & 0.7890 & 0.3547 \\
Dannon & 1.1410 & 0.1282 & 0.8291 & 0.3119 \\
Stonyfield Farm & 0.9037 & 0.1086 & 0.9519 & -0.0481 \\
Activia & 0.4919 & 0.1379 & 0.9360 & -0.4441 \\
\bottomrule
\end{tabular}
\end{table}

All costs are expected to be nonnegative; the cost of Activia should be approximately $0.1957$.

\subsection*{Question 23}
To simulate a merger between Chobani and Dannon while holding marginal costs fixed, we solve the pricing fixed point with an ownership matrix that sets the two products under common ownership. Prices of the merging products rise because the firm internalizes within-nest competition; rivals' prices typically increase by strategic complementarity. The table also reports the post-merger prices from the standard logit (Section 3); nested-logit post-merger prices are generally higher when $\hat\rho>0$.
\begin{table}[H]
\centering
\caption{Merger counterfactual (Chobani+Dannon), City 1, Period 1.}
\begin{tabular}{lcccc}
\toprule
Product & Old price & New price (NL merge) & New price (MNL merge) \\
\midrule
Yoplait & 0.803157 & 0.807493 & 0.809912 \\
Chobani & 1.151265 & 1.062118 & 1.207848 \\
Dannon & 1.145459 & 1.636437 & 1.219309 \\
Stonyfield Farm & 0.908994 & 0.911647 & 0.913530 \\
Activia & 0.498879 & 0.502368 & 0.504588 \\
\bottomrule
\end{tabular}
\label{tab:q23_nl_merge}
\end{table}


\subsection*{Question 24}
Per-capita consumer surplus in nested logit is
\[
CS(p)=\frac{1}{\hat\alpha}\log\!\Bigg(1+\sum_{g}\Big[\sum_{j\in g}e^{\delta^{obs}_j-\hat\alpha p_j\over 1-\hat\rho}\Big]^{1-\hat\rho}\Bigg).
\]
Using the pre- and post-merger fixed-point prices from Question 23, we compute the welfare change and compare it to per-capita expenditure $\sum_j p_js_j$. Nested logit predicts larger welfare losses than the standard logit because within-nest substitution is stronger and the merger internalizes more cannibalization.
\begin{table}[H]
\centering
\caption{Welfare before vs. after merger (City 1, Period 1).}
\begin{tabular}{lcc}
\toprule
Metric & Nested Logit & Standard Logit \\
\midrule
CS per capita & 1.234796 & 0.314144 \\
Total expend. per capita & 0.654330 & 0.631175 \\
Difference (post-pre) & -0.020828 & -0.031515 \\
CS change as % of pre expend. & -3.183035 & -4.993115 \\
\bottomrule
\end{tabular}
\label{tab:q24_nl_welfare}
\end{table}


\end{spacing}
\end{document}
